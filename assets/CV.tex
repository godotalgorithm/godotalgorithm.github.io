%%%%%%%%%%%%%%%%%%%%%%%%%%%%%%%%%%%%%%%%%%%%%%%%%%%%%%%%%%%%%%%%%%%%%%%%
%%%%%%%%%%%%%%%%%%%%%% Simple LaTeX CV Template %%%%%%%%%%%%%%%%%%%%%%%%
%%%%%%%%%%%%%%%%%%%%%%%%%%%%%%%%%%%%%%%%%%%%%%%%%%%%%%%%%%%%%%%%%%%%%%%%

%%%%%%%%%%%%%%%%%%%%%%%%%%%%%%%%%%%%%%%%%%%%%%%%%%%%%%%%%%%%%%%%%%%%%%%%
%% NOTE: If you find that it says                                     %%
%%                                                                    %%
%%                           1 of ??                                  %%
%%                                                                    %%
%% at the bottom of your first page, this means that the AUX file     %%
%% was not available when you ran LaTeX on this source. Simply RERUN  %%
%% LaTeX to get the ``??'' replaced with the number of the last page  %%
%% of the document. The AUX file will be generated on the first run   %%
%% of LaTeX and used on the second run to fill in all of the          %%
%% references.                                                        %%
%%%%%%%%%%%%%%%%%%%%%%%%%%%%%%%%%%%%%%%%%%%%%%%%%%%%%%%%%%%%%%%%%%%%%%%%

%%%%%%%%%%%%%%%%%%%%%%%%%%%% Document Setup %%%%%%%%%%%%%%%%%%%%%%%%%%%%

% Don't like 10pt? Try 11pt or 12pt
\documentclass[10pt]{article}

% This is a helpful package that puts math inside length specifications
\usepackage{calc}
\usepackage{etaremune}

% Simpler bibsection for CV sections
% (thanks to natbib for inspiration)
\makeatletter
\newlength{\bibhang}
\setlength{\bibhang}{1em}
\newlength{\bibsep}
 {\@listi \global\bibsep\itemsep \global\advance\bibsep by\parsep}
\newenvironment{bibsection}%
        {\vspace{-1.6\baselineskip}\begin{etaremune}[leftmargin=1.6em]{%
       \setlength{\leftmargin}{\bibhang}%
       \setlength{\itemindent}{-\leftmargin}%
       \setlength{\itemsep}{\bibsep}%
       \setlength{\parsep}{\z@}%
        \setlength{\partopsep}{0pt}%
        \setlength{\topsep}{0pt}}}
        {\end{etaremune}\vspace{-.6\baselineskip}}
\makeatother

% Layout: Puts the section titles on left side of page
\reversemarginpar

%
%         PAPER SIZE, PAGE NUMBER, AND DOCUMENT LAYOUT NOTES:
%
% The next \usepackage line changes the layout for CV style section
% headings as marginal notes. It also sets up the paper size as either
% letter or A4. By default, letter was used. If A4 paper is desired,
% comment out the letterpaper lines and uncomment the a4paper lines.
%
% As you can see, the margin widths and section title widths can be
% easily adjusted.
%
% ALSO: Notice that the includefoot option can be commented OUT in order
% to put the PAGE NUMBER *IN* the bottom margin. This will make the
% effective text area larger.
%
% IF YOU WISH TO REMOVE THE ``of LASTPAGE'' next to each page number,
% see the note about the +LP and -LP lines below. Comment out the +LP
% and uncomment the -LP.
%
% IF YOU WISH TO REMOVE PAGE NUMBERS, be sure that the includefoot line
% is uncommented and ALSO uncomment the \pagestyle{empty} a few lines
% below.
%

%% Use these lines for letter-sized paper
\usepackage[paper=letterpaper,
            %includefoot, % Uncomment to put page number above margin
            marginparwidth=1.2in,     % Length of section titles
            marginparsep=.05in,       % Space between titles and text
            margin=1in,               % 1 inch margins
            includemp]{geometry}

%% Use these lines for A4-sized paper
%\usepackage[paper=a4paper,
%            %includefoot, % Uncomment to put page number above margin
%            marginparwidth=30.5mm,    % Length of section titles
%            marginparsep=1.5mm,       % Space between titles and text
%            margin=25mm,              % 25mm margins
%            includemp]{geometry}

%% More layout: Get rid of indenting throughout entire document
\setlength{\parindent}{0in}

%% This gives us fun enumeration environments. compactitem will be nice.
\usepackage{paralist}

%% Reference the last page in the page number
%
% NOTE: comment the +LP line and uncomment the -LP line to have page
%       numbers without the ``of ##'' last page reference)
%
% NOTE: uncomment the \pagestyle{empty} line to get rid of all page
%       numbers (make sure includefoot is commented out above)
%
\usepackage{fancyhdr,lastpage}
\pagestyle{fancy}
%\pagestyle{empty}      % Uncomment this to get rid of page numbers
\fancyhf{}\renewcommand{\headrulewidth}{0pt}
\fancyfootoffset{\marginparsep+\marginparwidth}
\newlength{\footpageshift}
\setlength{\footpageshift}
          {0.5\textwidth+0.5\marginparsep+0.5\marginparwidth-2in}
\lfoot{\hspace{\footpageshift}%
       \parbox{4in}{\, \hfill %
                    \arabic{page} of \protect\pageref*{LastPage} % +LP
%                    \arabic{page}                               % -LP
                    \hfill \,}}

% Finally, give us PDF bookmarks
\usepackage{color,hyperref}
\hypersetup{colorlinks,breaklinks,
            linkcolor=blue,urlcolor=blue,
            anchorcolor=blue,citecolor=blue}

%%%%%%%%%%%%%%%%%%%%%%%% End Document Setup %%%%%%%%%%%%%%%%%%%%%%%%%%%%


%%%%%%%%%%%%%%%%%%%%%%%%%%% Helper Commands %%%%%%%%%%%%%%%%%%%%%%%%%%%%

% The title (name) with a horizontal rule under it
%
% Usage: \makeheading{name}
%
% Place at top of document. It should be the first thing.
\newcommand{\makeheading}[1]%
        {\hspace*{-\marginparsep minus \marginparwidth}%
         \begin{minipage}[t]{\textwidth+\marginparwidth+\marginparsep}%
                {\large \bfseries #1}\\[-0.15\baselineskip]%
                 \rule{\columnwidth}{1pt}%
         \end{minipage}}

% The section headings
%
% Usage: \section{section name}
%
% Follow this section IMMEDIATELY with the first line of the section
% text. Do not put whitespace in between. That is, do this:
%
%       \section{My Information}
%       Here is my information.
%
% and NOT this:
%
%       \section{My Information}
%
%       Here is my information.
%
% Otherwise the top of the section header will not line up with the top
% of the section. Of course, using a single comment character (%) on
% empty lines allows for the function of the first example with the
% readability of the second example.
\renewcommand{\section}[2]%
        {\pagebreak[2]\vspace{1.3\baselineskip}%
         \phantomsection\addcontentsline{toc}{section}{#1}%
         \hspace{0in}%
         \marginpar{
         \raggedright \scshape #1}#2}

% An itemize-style list with lots of space between items
\newenvironment{outerlist}[1][\enskip\textbullet]%
        {\begin{itemize}[#1]}{\end{itemize}%
         \vspace{-.6\baselineskip}}

% An environment IDENTICAL to outerlist that has better pre-list spacing
% when used as the first thing in a \section
\newenvironment{lonelist}[1][\enskip\textbullet]%
        {\vspace{-\baselineskip}\begin{list}{#1}{%
        \setlength{\partopsep}{0pt}%
        \setlength{\topsep}{0pt}}}
        {\end{list}\vspace{-.6\baselineskip}}

% An itemize-style list with little space between items
\newenvironment{innerlist}[1][\enskip\textbullet]%
        {\begin{compactitem}[#1]}{\end{compactitem}}

% An environment IDENTICAL to innerlist that has better pre-list spacing
% when used as the first thing in a \section
\newenvironment{loneinnerlist}[1][\enskip\textbullet]%
        {\vspace{-\baselineskip}\begin{compactitem}[#1]}
        {\end{compactitem}\vspace{-.6\baselineskip}}

% To add some paragraph space between lines.
% This also tells LaTeX to preferably break a page on one of these gaps
% if there is a needed pagebreak nearby.
\newcommand{\blankline}{\quad\pagebreak[2]}

% Uses hyperref to link DOI
\newcommand\doilink[1]{\href{http://dx.doi.org/#1}{#1}}
\newcommand\doi[1]{doi:\doilink{#1}}


%%%%%%%%%%%%%%%%%%%%%%%% End Helper Commands %%%%%%%%%%%%%%%%%%%%%%%%%%%

%%%%%%%%%%%%%%%%%%%%%%%%% Begin CV Document %%%%%%%%%%%%%%%%%%%%%%%%%%%%

\begin{document}
\makeheading{Jonathan E. Moussa}

\section{Contact Information}
%
% NOTE: Mind where the & separators and \\ breaks are in the following
%       table.
%
% ALSO: \rcollength is the width of the right column of the table
%       (adjust it to your liking; default is 1.85in).
%
\newlength{\rcollength}\setlength{\rcollength}{1.95in}%
%
\begin{tabular}[t]{@{}p{\textwidth-\rcollength}p{\rcollength}}
\href{https://molssi.org/}{Molecular Sciences Software Institute} & \textit{Cell:} (508) 498-1920 \\
 1880 Pratt Drive, Suite 1100 & \textit{E-mail:} \href{mailto:jemoussa@vt.edu}{jemoussa@vt.edu}  \\
 Blacksburg, VA 24060
\end{tabular}

\section{Citizenship}
%
USA

\section{Research Interests}
%
Computational physics, computational materials science, scientific computing, condensed matter physics,
 electronic structure theory, quantum chemistry, quantum information theory, quantum error correction, quantum simulation,
 numerical analysis, numerical linear algebra, classical algorithms, quantum algorithms.

\section{Education}
%
\href{http://www.berkeley.edu/}{\textbf{The University of California at Berkeley}}, Berkeley, CA, USA
\begin{outerlist}

\item[] Ph.D., \href{http://www.physics.berkeley.edu/} {Physics}, May 2008
        \begin{innerlist}
        \item Thesis Topic: \emph{Theoretical study of electron-phonon superconductivity}
        \item Adviser:
              \href{http://civet.berkeley.edu/cohen/mlcohen/}
                   {Professor Marvin L. Cohen}
        \end{innerlist}
\end{outerlist}

\blankline

\href{http://www.wpi.edu/}{\textbf{Worcester Polytechnic Institute}}, Worcester, MA, USA
\begin{outerlist}

\item[] M.S.,
        \href{http://www.wpi.edu/academics/Depts/Physics/}
             {Physics}, May 2004
        \begin{innerlist}
        \item Thesis Topic: \emph{The Schroedinger-Poisson selfconsistency in layered quantum semiconductor structures}
        \item Adviser:
              \href{http://users.wpi.edu/~lrram/}
                   {Professor L. Ramdas Ram-Mohan}
        \end{innerlist}

\item[] B.S.,
        \href{http://www.wpi.edu/academics/Depts/Math/}
             {Mathematics} and
        \href{http://www.wpi.edu/academics/Depts/Physics/}
             {Physics}, May 2001

\end{outerlist}

\section{Job Experience}
%
\textbf{Software Scientist} \hfill {July 2018 to present}
\begin{innerlist}

\item[] \href{https://molssi.org/}{Molecular Sciences Software Institute}, Blacksburg, VA USA
\end{innerlist}

\textbf{Staff Scientist} \hfill {November 2014 to July 2018}
\begin{innerlist}

\item[] {Nonconventional Computing Technologies Department},\\
        {Center for Computing Research},\\
        \href{http://www.sandia.gov/}{Sandia National Laboratories}
\end{innerlist}

\textbf{Postdoctoral Researcher} \hfill {August 2011 to November 2014}
\begin{innerlist}

\item[] {Advanced Device Technologies Department},\\
        {Center for Computing Research},\\
        \href{http://www.sandia.gov/}{Sandia National Laboratories}
\end{innerlist}

\textbf{Postdoctoral Researcher} \hfill {July 2008 to July 2011}
\begin{innerlist}

\item[] \href{http://tesla.ices.utexas.edu/}{Center for Computational Materials},\\
        \href{http://www.ices.utexas.edu/}{Institute for Computational Engineering and Sciences},\\
        \href{http://www.utexas.edu/}{The University of Texas at Austin}
\end{innerlist}

\textbf{Graduate Student} \hfill {August 2002 to June 2008}
\begin{innerlist}

\item[] \href{http://physics.berkeley.edu/}{Physics Department},\\
        \href{http://www.berkeley.edu/}{The University of California at Berkeley}
\end{innerlist}

\textbf{Summer Intern} \hfill {Summers of 2000 and 2002}
\begin{innerlist}

\item[] \href{http://www.wpafb.af.mil/afrl/ry}{Air Force Research Laboratory's Sensors Directorate}, \\
\href{http://www.wpafb.af.mil/}{Wright Patterson Air Force Base}
\end{innerlist}

\textbf{Programming Consultant} \hfill {August 1998 to June 2002}
\begin{innerlist}

\item[] Quantum Semiconductor Algorithms, Northborough, MA USA
\end{innerlist}

\section{H-index} \textbf{12} (\href{https://scholar.google.com/citations?user=2w3785cAAAAJ&hl=en&oi=ao}{Google Scholar})

\section{Publications} \begin{bibsection}

    \item \textbf{Moussa, J. E.} and A. D. Baczewski. Assessment of localized and randomized algorithms for electronic structure.
    \emph{Electronic Structure} 1:033001 (2019). \doi{10.1088/2516-1075/ab2022}.

    \item Hardy, W. J., C. T. Harris, Y.-H. Su, Y. Chuang, \textbf{J. Moussa}, L. N. Maurer, J.-Y. Li, T.-M. Lu, and D. R. Luhman. Single and double hole quantum dots in strained Ge/SiGe quantum wells.
    \emph{Nanotechnology} 30:215202 (2019). \doi{10.1088/1361-6528/ab061e}.

    \item Chou, C.-T.,  N. T. Jacobson, \textbf{J. E. Moussa},  A. D. Baczewski,  Y. Chuang, C.-Y. Liu,  J.-Y. Li, and T. M. Lu. Weak anti-localization of two-dimensional holes in germanium beyond the diffusive regime.
    \emph{Nanoscale} 10:20559 (2018). \doi{10.1039/C8NR05677C}.

    \item \textbf{Moussa, J. E.}. Minimax rational approximation of the Fermi-Dirac distribution.
    \emph{The Journal of  Chemical Physics} 145:164108 (2016). \doi{10.1063/1.4965886}.

    \item \textbf{Moussa, J. E.}. Transversal Clifford gates on folded surface codes.
    \emph{Physical Review A} 94:042316 (2016). \doi{10.1103/PhysRevA.94.042316}.

    \item \textbf{Moussa, J. E.}. Quantum circuits for qubit fusion.
    \emph{Quantum Information and Computation} 16:1113 (2016). \href{https://arxiv.org/abs/1512.06132}{arXiv:1512.06132}.

    \item Parekh, O., J. Wendt, L. Shulenburger, A. Landahl, \textbf{J. Moussa}, and J. Aidun. Benchmarking Adiabatic Quantum Optimization for Complex Network Analysis.
    \emph{Journal of Intelligence Community Research and Development} (2015). \\ \href{https://arxiv.org/abs/1604.00319}{arXiv:1604.00319}.

    \item Gamble, J. K., N. T. Jacobson, E. Nielsen, A. D. Baczewski, \textbf{J. E. Moussa}, I. Monta\~{n}o, and R. P. Muller. Multivalley effective mass theory simulation of donors in silicon.
     \emph{Physical Review B}, 91:235318 (2015). \doi{10.1103/PhysRevB.91.235318}.

    \item Chandra, R., N. T. Jacobson, \textbf{J. E. Moussa}, S. H. Frankel, and S. Kais. Quadratic constrained mixed discrete optimization with an adiabatic quantum optimizer.
     \emph{Physical Review A}, 90:012308 (2014). \doi{10.1103/PhysRevA.90.012308}.

    \item \textbf{Moussa, J. E.}. Cubic-scaling algorithm and self-consistent field for the random-phase approximation with second-order screened exchange.
     \emph{The Journal of Chemical Physics}, 140:014107 (2014). \doi{10.1063/1.4855255}.

    \item \textbf{Moussa, J. E.}, S. M. Foiles, and P. A. Schultz.
         Simulation and modeling of the electronic structure of GaAs damage clusters.
         \emph{Journal of Applied Physics}, 113:093706 (2013). \doi{10.1063/1.4794164}.

    \item \textbf{Moussa, J. E.}.
         Comment on ``Fast and Accurate Modeling of Molecular Atomization Energies with Machine Learning''.
         \emph{Physical Review Letters}, 109:059801 (2012). \doi{10.1103/PhysRevLett.109.059801}.

    \item \textbf{Moussa, J. E.}, P. A. Schultz, and J. R. Chelikowsky.
         Analysis of the Heyd-Scuseria-Ernzerhof density functional parameter space.
         \emph{The Journal of Chemical Physics}, 136:204117 (2012). \doi{10.1063/1.4722993}.

    \item \textbf{Moussa, J. E.}, N. Marom, N. Sai, and J. R. Chelikowsky.
         Theoretical Design of a Shallow Donor in Diamond by Lithium-Nitrogen Codoping.
         \emph{Physical Review Letters}, 108:226404 (2012). \doi{10.1103/PhysRevLett.108.226404}.

    \item Marom, N., \textbf{J. E. Moussa}, X. Ren, A. Tkatchenko, and J. R. Chelikowsky.
         Electronic structure of dye-sensitized TiO$_2$ clusters from many-body perturbation theory.
         \emph{Physical Review B}, 84:245115 (2011). \doi{10.1103/PhysRevB.84.245115}.

    \item Marom, N., X. Ren, \textbf{J. E. Moussa}, J. R. Chelikowsky, and L. Kronik.
         Electronic structure of copper phthalocyanine from $G_0 W_0$ calculations.
         \emph{Physical Review B}, 84:195143 (2011). \doi{10.1103/PhysRevB.84.195143}.

    \item \textbf{Moussa, J. E.}, J. Noffsinger, and M. L. Cohen.
         Possible thermodynamic stability and superconductivity of Be$_2$B$_x$C$_{1-x}$.
         \emph{Physical Review B}, 78:104506 (2008). \doi{10.1103/PhysRevB.78.104506}.

    \item \textbf{Moussa, J. E.} and M. L. Cohen.
         Using molecular fragments to estimate electron-phonon coupling and possible superconductivity in covalent materials.
         \emph{Physical Review B}, 78:064502 (2008). \doi{10.1103/PhysRevB.78.064502}.

    \item \textbf{Moussa, J. E.} and M. L. Cohen.
         Constraints on $T_c$ for superconductivity in heavily boron-doped diamond.
         \emph{Physical Review B}, 77:064518 (2008). \\ \doi{10.1103/PhysRevB.77.064518}.

    \item \textbf{Moussa, J. E.} and M. L. Cohen.
         Two bounds on the maximum phonon-mediated superconducting transition temperature.
         \emph{Physical Review B}, 74:094520 (2006). \\ \doi{10.1103/PhysRevB.74.094520}.

    \item Ram-Mohan, L. R., K. H. Yoo, and \textbf{J. Moussa}.
         The Schr\"odinger-Poisson self-consistency in layered quantum semiconductor structures.
         \emph{Journal of Applied Physics}, 95:3081 (2004). \doi{10.1063/1.1649458}.

    \item \textbf{Moussa, J.}, L. R. Ram-Mohan, A. C. H. Rowe, and S. A. Solin.
        Response of an extraordinary magnetoresistance read head to a magnetic bit.
        \emph{Journal of Applied Physics}, 94:1110 (2003). \doi{10.1063/1.1576897}.

    \item \textbf{Moussa, J.}, L. R. Ram-Mohan, J. Sullivan, T. Zhou, D. R. Hines, and S. A. Solin.
        Finite-element modeling of extraordinary magnetoresistance in thin film semiconductors with metallic inclusions.
        \emph{Physical Review B}, 64:184410 (2001). \doi{10.1103/PhysRevB.64.184410}.

\end{bibsection}

\section{Unpublished Work} \begin{bibsection}

    \item \textbf{Moussa, J. E.}. Minimax separation of the Cauchy kernel.
    \href{https://arxiv.org/abs/1909.06911}{arXiv:1909.06911} (2019).

    \item Metcalf, M., \textbf{J. E. Moussa}, W. A. de Jong, M. Sarovar. Engineered thermalization of quantum many-body systems.
    \href{https://arxiv.org/abs/1909.02023}{arXiv:1909.02023} (2019).

    \item \textbf{Moussa, J. E.}. Measurement-Based Quantum Metropolis Algorithm.
    \href{https://arxiv.org/abs/1903.01451}{arXiv:1903.01451} (2019).

   \item Shulenburger, L., A. D. Baczewski, S. M. Foiles, A. E. Wills, N. A. Modine, \textbf{J. E. Moussa}, P. A. Schultz, V. Tikare, and A. F. Wright.
   Next-Generation Electronic Structure Codes. Sandia Technical Report SAND2016-9782 (2016).

    \item \textbf{Moussa, J. E.}. Linear embedding of free energy minimization.
    \href{https://arxiv.org/abs/1603.05180}{arXiv:1603.05180} (2016).

    \item Metodi, T. S., A. J. Landahl, C. Ryan-Anderson, M. S. Carroll, \textbf{J. E. Moussa}, and R. P. Muller.
    SEQIS Late Start LDRD: Final Report - Robust Quantum Operations. Sandia Technical Report SAND2015-10754 (2015).

    \item \textbf{Moussa, J. E.} and A. D. Baczewski. Comment on ``Self-Averaging Stochastic Kohn-Sham Density-Functional Theory''.
     \href{http://arxiv.org/abs/1311.6576}{arXiv:1311.6576} (2013).

    \item \textbf{Moussa, J. E.}. Comment on ``Adiabatic Quantum Algorithm for Search Engine Ranking''.
     \href{http://arxiv.org/abs/1310.6676}{arXiv:1310.6676} (2013).

    \item \textbf{Moussa, J. E.} and P. A. Schultz. Accurate numerical integration of an electron exchange hole with a screened Coulomb interaction.
     \href{http://arxiv.org/abs/1210.8233}{arXiv:1210.8233} (2012).
     
    \item \textbf{Moussa, J. E.}. Generalized unitary Bogoliubov transformation that breaks fermion number parity.
    \href{https://arxiv.org/abs/1208.1086}{arXiv:1208.1086} (2012).

    \item \textbf{Moussa, J. E.}. Approximate diagonalization method for many-fermion Hamiltonians. \href{http://arxiv.org/abs/1003.2596}{arXiv:1003.2596} (2010).
 
    \item \textbf{Moussa, J. E.}. Perfect algebraic coarsening. \href{http://arxiv.org/abs/math/0505157}{arXiv:math/0505157} (2005).

\end{bibsection}

\section{Awarded Grants} \begin{bibsection}

    \item \textbf{Moussa, J. E.} and M. Sarovar. Realizing the Power of Near-Term Quantum Technologies. 0.6 FTE for 2 years.
    \emph{Sandia National Labs' Laboratory Directed Research and Development Program, Defense Systems and Assessments Mission Area} (2015).

\end{bibsection}

\section{Invited Talks} \begin{bibsection}

  \item \textbf{Moussa, J. E.}. Local reduction of Hermitian eigenproblems. \emph{2017 Meeting of the International Linear Algebra Society}
  (July 24, 2017).

  \item \textbf{Moussa, J. E.}. Designing shallow donors in diamond. \emph{March Meeting of the American Physical Society}
  (March 4, 2015).

  \item \textbf{Moussa, J. E.}. Interdisciplinary Perspectives on Electronic Structure Theory. \emph{Scuseria research group, Rice University}
  (January 8, 2014).

  \item \textbf{Moussa, J. E.}. Eigensolvers in Condensed Matter Physics, Condensed Matter Physics in Eigensolvers. \emph{Computer Science Research Institute, Sandia National Laboratories}
  (March 29, 2011).
  
  \item Chelikowsky, J. R. and \textbf{J. E. Moussa}. Algorithms for the Quantum Modeling of the Properties of Nanocrystals, Nanofilms, and Nanowires.
   \emph{SIAM Conference on Computational Science and Engineering} (March 4, 2009).
  
\end{bibsection}

\section{Contributed Talks} \begin{bibsection}

 \item \textbf{Moussa, J. E.}. Testing, analysis, and refinement of the quantum Metropolis algorithm. \emph{March Meeting of the American Physical Society}
  (March 5, 2019).

 \item \textbf{Moussa, J. E.}.  Localized and Randomized Algorithms for Electronic Structure. \emph{March Meeting of the American Physical Society}
  (March 7, 2018).

 \item \textbf{Moussa, J. E.}. Transversal Clifford gates on folded surface codes. \emph{March Meeting of the American Physical Society}
  (March 16, 2017).

 \item \textbf{Moussa, J. E.}. Convex Lower Bounds for Free Energy Minimization. \emph{March Meeting of the American Physical Society}
  (March 16, 2016).

 \item \textbf{Moussa, J. E.}. Quantum Simulation: Classical Algorithms Versus Analog Simulators. \emph{March Meeting of the American Physical Society}
  (March 3, 2015).

 \item \textbf{Moussa, J. E.}. Maximum entropy quantum simulation. \emph{XXVI IUPAP Conference on Computational Physics}
  (August 13, 2014).

 \item \textbf{Moussa, J. E.}. Acceleration of screened-exchange density-functional calculations with approximate differential overlap. \emph{March Meeting of the American Physical Society}
  (March 4, 2014).

 \item \textbf{Moussa, J. E.}. Cubic-scaling algorithm and self-consistent mean field for the random-phase approximation with second-order screened exchange. \emph{March Meeting of the American Physical Society}
  (March 18, 2013).

 \item \textbf{Moussa, J. E.}, P. A. Schultz, and J. R. Chelikowsky. Retrofit of the HSE density functional. \emph{March Meeting of the American Physical Society} (February 27, 2012).

 \item \textbf{Moussa, J. E.} and J. R. Chelikowsky. Size-dependence of electronic and optical properties of armchair graphene nanoislands. \emph{March Meeting of the American Physical Society} (March 23, 2011).

 \item \textbf{Moussa, J. E.} and J. R. Chelikowsky. The Truncated Eigenfermion Decomposition applied to the Hubbard model. \emph{March Meeting of the American Physical Society} (March 15, 2010).

 \item \textbf{Moussa, J. E.} and J. R. Chelikowsky. A Brief Introduction to the Truncated Eigenfermion Decomposition. \emph{March Meeting of the American Physical Society} (March 17, 2009).

 \item \textbf{Moussa, J. E.} and M. L. Cohen. Constraints on $T_c$ for superconductivity in heavily boron-doped diamond. \emph{March Meeting of the American Physical Society} (March 11, 2008).

 \item \textbf{Moussa, J. E.} and M. L. Cohen. Possibility of superconductivity in high pressure phases of BC$_3$. \emph{March Meeting of the American Physical Society} (March 6, 2007).

 \item \textbf{Moussa, J. E.} and M. L. Cohen. Stability Constraints and Local Criteria for the Bounds on $T_c$ of Conventional Superconductors. \emph{March Meeting of the American Physical Society} (March 16, 2006).

 \item \textbf{Moussa, J. E.} and M. L. Cohen. Compact representation of the Green function of an infinite periodic system. \emph{March Meeting of the American Physical Society} (March 23, 2005).

\end{bibsection}

\section{Patents} \begin{bibsection}
    \item Hine, D. R., S. A. Solin, T. Zhou, \textbf{J. E. Moussa}, L. R. Ram-Mohan, J. M. Sullivan Jr.
          Method and system for finite element modeling and simulation of enhanced magnetoresistance in thin film semiconductors with metallic inclusions. U.S. Patent \#6,937,967 (2005).
\end{bibsection}

\section{Referee for Journals} \begin{loneinnerlist}
    \item \emph{Physical Review B}
    \item \emph{Physical Review Letters}
    \item \emph{The Journal of Chemical Physics}
    \item \emph{Mathematical Reviews}
\end{loneinnerlist}

\section{Awards} \begin{loneinnerlist}
%
    \item \emph{Salisbury Prize}, Worcester Polytechnic Institute (2001).

    \item \emph{Employee Recognition Award}, Enceladus: Quantum Computer Benchmarking Team, Sandia National Laboratories (2016).
\end{loneinnerlist}

\section{Teaching Experience}
\href{http://www.berkeley.edu}{\textbf{The University of California at Berkeley}},
Berkeley, CA USA
\begin{outerlist}

\item[] \textit{Graduate Student Instructor}%
    \hfill \textbf{September 2002 to December 2002}
\end{outerlist}

\section{Programming Skills}
%
 C, C++, Fortran, Python, \TeX{}

\end{document}

%%%%%%%%%%%%%%%%%%%%%%%%%% End CV Document %%%%%%%%%%%%%%%%%%%%%%%%%%%%%
